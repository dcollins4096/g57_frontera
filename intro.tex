
We are requesting \request\ on Frontera in order to perform several studies of
the interstellar medium (ISM).  This is part of an effort to observe
gravitational waves from the big bang, imprinted in the cosmic microwave background (CMB).  These observations are hampered by the
dust and plasma in our own galaxy, which outshines the CMB.  We will simulate
the plasma in the ISM in two ways to characterize its observational properties.
One study will be high resolution simulations of plasma turbulence, and the
other suite will be simulations of isolated galaxies.

\section{Background}

Gravitational waves generated at
the beginning of the Universe leave an imprint in the polarization of the cosmic
microwave background (CMB).  The CMB is the oldest light observable in the
Universe, and gives us a snapshot of its very early stages.
Observing the polarization of the CMB will give us a measurement of
gravitational waves that were produced at the formation of the Universe.
Unfortunately, the dusty plasma in our own Milky Way also produces polarized
light that is brighter than the primordial signal.  Our simulations will study
the polarization properties of the light from the plasma in Milky Way type
galaxies.  Understanding this polarization is essential for its removal from
future measurements of the CMB.

The Planck satellite measured the polarization of the whole sky.  Polarization,
having direction on the sky, needs two quantities for its description. The most
useful option are the parity-even $E$-mode and parity-odd $B$-mode.  
The only source of $B-$ mode polarization in the CMB are gravitational waves.
So searching for $B-$modes in the CMB will prove quite profitable.  However, the
dusty, magnetized plasma and synchrotron electrons in our own galaxy produce
$B-$mode polarization.  So this must be understood so it can be removed.  

The Planck satellite measured the $E$- and $B$-mode signals over the entire sky.
They found that both are relatively uniform power-laws in wavenumber, with power
spectra given by powerlaws
\begin{align}
C_\ell^{EE} = A^{EE} k^{\alpha^{EE}},
\end{align}
for $E-$mode and a similar expression for $B-$mode.  They found that the slopes
are roughly equal, $\alpha^{EE}\sim\alpha^{BB}\sim-2.5$, and the $B-$mode have
roughly half the power of the $E-$mode.  

We have shown (Stalpes et al 2023, in prep) that the slopes of the polarization
are directly related to the Mach number, $\mach=v/c_s$, the ratio of
r.m.s velocity to sound speed, and \alf\ mach number, $\alfmach=v/v_A$, where
$v_A$ is the speed of magnetic disturbances.  Increasing \mach\ makes for more
filamentary structure in the cloud, and slopes in line with what Planck
observed.  We ran a suite of \mach = (\half,1,2,3,4,5,6) and \alfmach=(\half,1,2).  After
interpolating, we predict that the observed values of $\alpha^{EE}$ and
$\alpha^{BB}$ are best matched by plasma with \mach=4.7 and \alfmach=1.5. The
study presented in Stalpes et al were run at a modest resolution of $512^3$.  

Another curious finding in the Planck data is a correlation between the total
signal, $T-$mode, which is parity-even, and the parity-odd $B-$mode.  In principle, these should be
uncorrelated, but the Planck satellite measured a non-zero correlation at the
5\% level.  The results from Stalpes et al (2023) show that turbulence alone can
account for such a correlation only at the 2\% level at best, and these were
relatively low resolution studies.  It is entirely likely that the $TB$
correlation comes from large scale features in the dust and magnetic field
morphology.  To properly reproduce the large scale magnetic field, we will
simulate an entire Milky Way sized galaxy as well as the circumgalactic medium
(CGM) that surrounds it. The dust that we are interested in, as well as the star
formation and explosions that determine the energetics of the galaxy, all happen
in a thin disk, roughly 100pc thick and 20,000pc across.  Plasma is ejected from
the disk by supernovae, and expelled into the CGM, where it cools and falls back
to the disk.  This baryon cycling dictates the star formation activity in the
galaxy and impacts the magnetic field strength and morphology throughout the
system.  All of these pieces are important for a complete picture of the
polarized sky.

We propose two suites of simulations.  The first is a continuation of the
moderate resolution turbulent boxes presented in Stalpes et al (2023).  In that
study, we predict that a Mach number of 4.7 and an \alf\ Mach number of 1.5
reproduce the sky.  We will perform two simulations at $2048^3$.  These
simulations will test our prediction, and provide high quality datasets that
reproduce the salient features of the sky that can be used to test foreground
removal algorithms algorithms and ISM models.  While these simulations are quite
large, the excellent scaling and our years of experience guarantee success.
The second suite of simulations
is a set of three isolated galaxies that will simultaneously provide high
resolution on the mid plane, and properly resolve the environment and boundary
conditions of the galaxy.  These will be used initially to test CMB foreground
models, and will also have many applications beyond the CMB.  These galaxies
will be a stack of nine fixed resolution levels at roughly $512^3$ per level.
