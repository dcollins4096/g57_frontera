\section{Simulations}
\label{sec.simulations}

We will perform two suites of simulations, a suite of driven \emph{turbulence}
simulations and a suite of isolated \emph{galaxy} simulations.

\subsection{Turbulence}

\def\vrms{\ensuremath{V_{r.m.s}}}

In a turbulent system, such as the ISM or stirring cream in one's coffee,
energy is injected at a scale, $L$, and the fluid dynamics shreds that energy
into a cascade of ever decreasing size scale until the dissipation length is
reached.  The dissipation length scale, $\ell_D\propto \mach^{-3/4}$,
decreases as the mean kinetic energy increases.  There are three primary regimes
in a turbulent system; the large scale driving, the intermediate scale \emph{
inertial range},
and the small scale dissipation.  The driving scale is a boundary condition, and
the dissipation is done by the numerics.  
The inertial range is where simulations can teach us about nature, as it does
not depend on the user-supplied boundary conditions.  In simulations, the inertial range is
proportional the linear resolution.    We have performed a suite of simulations
of the ISM at a resolution of $512^3$, and now wish to build on that success
with two simulations at $2048^3$.

By using these linear relations found in our preliminary work,
 we predict that the ISM has a ``typical'' value of
$\mach=4.7$ and $\alfmach=1.5$.  This prediction was done at low resolution with
limited inertial range.  
We will further test this prediction against high resolution simulations, which
further separate the boundary condition and numerical shortcomings from the
signal of interest.

The actual ISM is filled with many
clouds with a range of Mach numbers, so we will explore this predicted
``typical'' cloud as well as one additional cloud at higher Mach number.  
These two simulations aim to maximize the useful portion of the
inertial range.  The first simulation will be at the typical $\mach=4.7$ and
$\alfmach=1.5$, and the second will be at $\mach=8$ and $\alfmach=1.5$.  This
higher Mach number is challenging to do at moderate resolution, as the length of
the shocks and the dissipation scale gets close to the grid resolution.  
The temptation to repeat the whole suite of 21 simulations is large, especially
considering the poor inertial range of the lower Mach number runs; however this
would be quite expensive in terms of disk and SU usage.

The simulations will be run for long enough to gather meaningful statistics.
The relevant timescale is the dynamical time, $t_{dyn}=L/\mach$, where $L$ is
the size of the velocity pattern, $L=1/2$.  This gives the time it takes for a
typical eddy to pass over itself.  We will run for 5$t_{dyn}$.  It takes about $2 t_{dyn}$ to establish a
steady-state as the dissipation structures are created by the cascade.  We will
then run for 3 additional dynamical times to obtain a statistically well sampled
set.  While more time would be desirable, this should be sufficient to obtain
meaningful results.

The size of these simulations will make them somewhat challenging to perform and
analyze.  Our recent successes outlined in the Progress document demonstrates
that we are quite ready for this scale, and the Scaling document shows that Enzo
will perform admirably.  

\subsection{Galaxy}

The next generation of CMB telescopes will reach an angular resolution of
$1^\circ$ in the near future, and $1^{\prime\prime}$ by the end of the mission.
This translates to linear size of (1.7pc, 0.03pc) at 100pc from the
observer.  Our near term stretch goal is to create a galaxy with 1.7pc
resolution for the dust at the mid-plane of the galaxy, while
also including the CGM at 1Mpc.  This is somewhat beyond our reach at the
moment, as an next step we will aim for 6.25pc resolution at the disk
and $0.8$Mpc for the outer box.  We will do
this with a stack of 8 levels of static AMR, with approximately $512^3$ zones on each
level.  We will quantize our resolution into grids of $32^3$ zones, rather than
the dynamic allocation of grids that is Enzo's speciality, in order to optimize
memory usage and performance.  

A spiral galaxy is essentially a spinning pancake of gas and stars in a ball of dark
matter and hot gas.   Thus our finest level will be wide and
flat ($100\rm{pc}\times 20,000 \rm{pc} \times 20,000 \rm{pc}$) at 6.25pc
resolution.  The next two level increase the aspect ratio of the refined region,
increasing a factor of 3 in height and about 20\% in width.  The remaining
levels will be cubic regions of $512^3$.  We will quantize our refinement
regions into blocks of $32^3$ in order to optimize the performance of the
simulation.

We will simulate for a total of 1Gyr.  A galaxy of this size has an orbital
period of about 250Myr, this gives 8 rotations of the galaxy.  The initial
conditions take about 500Myr to reach a roughly steady state; stellar explosions
keep the galaxy from collapsing to the mid-plane, but take a few Myr to happen.  
We will then give the galaxy several orbital times to establish the magnetic
field morphology and component distribution.  

We will run three simulations.  The first will be our unmagnetized fiducial run;  the second will
have a weak uniform magnetic field, and the third will have supernova deliver
magnetic energy.  From these we will examine the development of the large scale
magnetic field, its impact on the star formation behavior, and synthetic
observations from within for use in CMB foreground removal.  

This full 8 level AMR structure has been tested on Frontera, and found to perform as
expected from the scaling study.  The scaling study was restricted to one level
of refinement, as this is sufficient to engage all of the AMR overhead.  

These simulations will be extremely useful for the study of the CMB foregrounds
and star formation, and will also serve as a proving ground for our target of
1pc resolution.  The next step will be to repeat this work with $1024^3$ zones
per level, which will give 3pc resolution.  This will require some optimization
of the code and layout to be performant enough, but is well within our reach.  With $2048^3$ zones per level,
we will achieve 1pc resolution, though the expected cost of 100 million SU may
prove prohibitive.  

