\section{Request Details}


\label{sec.request}
The cost for each simulation is found as
\begin{align}
SU &= t_{wall} N_N\\
t_{wall} &= \frac{\sum_{\ell} N_Z N_U}{\zeta} \frac{1}{N_C},
\end{align}
where $SU$ is the total cost; $t_{wall}$ is the run time; $N_N$ is the number of
nodes; $N_Z$ is the number of zones per level $\ell$; $N_U$ is the number of
updates per level; $\zeta$ is
the performance measured in (zone-updates)/(processor-second); and $N_C$ is the
total number of cores.  The cost, $zeta$, is determined in the scaling document.
Total disk usage is found as
\begin{align}
	Disk = 4 N_Z N_F N_D \rm{bytes},
\end{align}
where each of $N_Z$ zones contains $N_F$ fields and we store $N_D$ dumps for
each run.  The values of $N_N$, $N_C$, $N_F$, $N_D$ and $\zeta$ can be found in
Table \ref{table1}.  The values of $N_Z$ and $N_U$ are outlined in Table
\ref{table2}.  We will outline the process for estimating $N_Z$ and $N_U$.


\begin{table}[h]
\begin{center}
\caption{The request broken down by simulation.  For the \emph{turbulence}
simulations, the duration, $T$ and time step,
$\Delta t$, are determined by \mach\ and \alfmach.  For the \emph{galaxy}
simulations, the AMR structure will be constructed in layers of grids with
$32^3$ zones each centered on the midplane of the galaxy. The outer box, level
0, is
$8.2\sci{5}$pc on a side, and the finest level has a resolution of 6.25 pc.
For all simulations, the number of updates for each level is $T/\Delta t$.  The
total cost $SU=N_Z * T/\Delta t / \zeta / N_C * N_N$, and
The total disk is $N_Z*N_F*8$ bytes, where $N_F$ is the number of fields.
For the \emph{turbulence} simulations, $N_C=32,768$, $N_N=586$, $N_F=20$, and
$\zeta=4\sci{5}$.  For the \emph{galaxy} simulations, $N_C=512$, $N_N=32$
$N_F=27$, and
$\zeta = 10^4$.  Details can be found in the main text. }
 
\label{table1}                                                                                                                                                               
\begin{tabular}{rrrrrr}
Suite & $N_N$ & $N_C$ & $N_F$ & $N_D$ & $\zeta$ \\
\hline
\emph{turbulence} & 586 & 32,768 & 17 &10 & $4\sci{5}$\\
\emph{galaxy} & 32 & 512 & 27 & 20 & $2\sci{5}$\\
\hline
\end{tabular}                                                                                                                                                               
\end{center}
\end{table}                                                                                                                                                                




\begin{table} \begin{center} \caption{The request broken down by simulation.  For the \emph{turbulence}
simulations, the duration, $T$ and time step,
$\Delta t$, are determined by \mach\ and \alfmach.  For the \emph{galaxy}
simulations, the AMR structure will be constructed in layers of grids with
$32^3$ zones each centered on the midplane of the galaxy. The outer box, level
0, is
$8.2\sci{5}$pc on a side, and the finest level has a resolution of 6.25 pc.
For all simulations, the number of updates for each level is $T/\Delta t$.  The
total cost $SU=N_Z * T/\Delta t / \zeta / N_C * N_N$, and
The total disk is $N_Z*N_F*8$ bytes, where $N_F$ is the number of fields.
For the \emph{turbulence} simulations, $N_C=32,768$, $N_N=586$, $N_F=20$, and
$\zeta=4\sci{5}$.  For the \emph{galaxy} simulations, $N_C=512$, $N_N=32$
$N_F=27$, and
$\zeta = 10^4$.  Details can be found in the main text. }
 \label{table2}                                                                                                                                                               
\begin{tabular}{               l               r               r                                               r               r               r               r              }
                  \hline                                                                                                                                               
                   suite       &$(\mach, \alfmach)$       &       &                                             $T$       &$\Delta t$       &   $N_Z$       &      SU      \\       
                  \hline                                                                                                                                               
                \emph{turbulence}       &(4.7, 1.5)       &       &                                            0.53       &1.1\sci{-6}       &$2048^3$       &5.3\sci{4}      \\       
                \emph{turbulence}       &(8, 1.5)       &       &                                            0.31       &6.6\sci{-7}       &$2048^3$       &5.0\sci{4}      \\       
                  \hline                                                                                                                                               
                               &               &       &                                                       &Turbulence: Disk       &1.3\sci{4}Gb & SU:1.0\sci{5}      \\       
                  \hline                                                                                                                                               
                   Suite      &    Level       &grid layout       &                                $T$[Gyr]       &$\Delta t$[Gyr]       &   $N_Z$       &      SU      \\       
                  \hline                                                                                                                                               
                \emph{galaxy}       &       8      &$     100  \times     100  \times       1      $&       2       &2.1\sci{-3}    & $(     689   )^3$&5.5\sci{5}      \\       
                \emph{galaxy}       &       7      &$      54  \times      54  \times       3      $&       2       &4.1\sci{-3}    & $(     659   )^3$&2.4\sci{5}      \\       
                \emph{galaxy}       &       6      &$      30  \times      30  \times      11      $&       2       &8.3\sci{-3}    & $(     687   )^3$&1.4\sci{5}      \\       
                \emph{galaxy}       &       5      &$      16  \times      16  \times      16      $&       2       &1.7\sci{-2}    & $(     512   )^3$&2.8\sci{4}      \\       
                \emph{galaxy}       &       4      &$      16  \times      16  \times      16      $&       2       &3.3\sci{-2}    & $(     512   )^3$&1.4\sci{4}      \\       
                \emph{galaxy}       &       3      &$      16  \times      16  \times      16      $&       2       &6.6\sci{-2}    & $(     512   )^3$&7.0\sci{3}      \\       
                \emph{galaxy}       &       2      &$      16  \times      16  \times      16      $&       2       &1.3\sci{-1}    & $(     512   )^3$&3.5\sci{3}      \\       
                \emph{galaxy}       &       1      &$      16  \times      16  \times      16      $&       2       &2.7\sci{-1}    & $(     512   )^3$&1.8\sci{3}      \\       
                \emph{galaxy}       &       0      &$      16  \times      16  \times      16      $&       2       &5.3\sci{-1}    & $(     512   )^3$&8.8\sci{2}      \\       
                  \hline                                                                                                                                               
                               &               &       &                                                       &One Galaxy: Disk       &7.0\sci{3}Gb & SU:9.8\sci{5}      \\       
                               &               &       &                                                       &Three Galaxies: Disk       &2.1\sci{4}Gb & SU:2.9\sci{6}      \\       
                  \hline                                                                                                                                               
                               &               &       &                                                       &Total: Disk       &3.4\sci{4}Gb & SU:3.0\sci{6}      \\       
\end{tabular}                                                                                                                                                               
\end{center}                                                                                                                                                               
\end{table}                                                                                                                                                                

The number of zones, $N_Z$, is computed from the problem geometry.  All of these
simulations will use fixed or static resolution, so $N_Z$ is known.  
For the \emph{turbulence} simulations, $N_Z=2048^3$.  For the galaxy
simulations, $N_Z$ will be computed level-by-level from the grid layout.  We construct each level in grid patches of $32^3$ zones each.  
Consistent patch size allows us to optimize the memory usage and performance of the simulation.
The finest level is a thin pancake of
$100\times100\times1$ grids. This covers the midplane with a
resolution of 6.25 pc for 
the entire disk, to the dust scale height of $\pm 100$pc and a side length of
20kpc.
Resolution increases by a factor of 2 each level.  The next two levels increase
the aspect ratio of the refined region ($54\times54\times3$ and
$30\times30\times11$ grids, respectively), and the remaining levels are $16\times 16 \times 16$
grids.   This grid structure gives an outer size of $8.2\sci{5}$pc on a side
with 6.25pc resolution over the whole disk at the midplane.  

The number of updates is found from the total simulation time over the step
size, $N_U=T/\Delta t$. $T$ by the
problem goals, and we estimate $\Delta t$ from the solver behavior and prior
results.
As described above, the \emph{turbulence} simulations will run for
 $T=5 t_{dyn} = 5/(2\mach)$ (in dimensionless code units), to statistically
resolve the turbulence. The \emph{galaxy} simulations will run for $T=2$ Gyr,
which represents roughly 8 orbits of the galaxy, hopefully enough time to grow
the magnetic field.  The timestep size, $\Delta t$,
is found by a standard Courant condition,
\begin{align}
	\Delta t = \eta \frac{\Delta x}{(v+c_f)_{max}} \label{eqn.dt},
\end{align}
where $(v+c_f)_{max}$ is the maximum signal velocity on each resolution level.  Here, $v$ is the velocity
and $c_f$ is the fast MHD speed.  It is not possible to predict the exact value
of this signal speed, so we calibrate to lower resolution simulations and use
Equation \ref{eqn.dt} to rescale.  For the \emph{turbulence} simulations, we
calibrate $\Delta t$ to the fiducial simulations in Stalpes et al (2023).  For
the \emph{galaxy} simulations, we calibrate to a low-resolution preliminary
galaxy.

Table \ref{table2} shows a summary of the request.  The first two rows show each
of  the \emph{turbulence} simulations.  The remaining rows outline the
\emph{galaxy} simulations, estimating $N_Z$ and $N_U=T/\Delta t$ and the total
cost, $SU$.  Also shown in the table is the projected disk usage for the new
simulations, as well as our archive already on Ranch of recently run
hydrodynamical turbulence simulations and star formation simulations.
